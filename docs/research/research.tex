\documentclass[a4paper]{article}
\usepackage{url}

\begin{document}
    I am currently researching how to find the BPM of a song below are some formula I was recommended.
    \section{On-Set Detection}
        \subsection{Sound energy algorithm}
            From link below \\
            The algorithm divides the data into blocks of samples and compares the energy of a block
            with the energy of a preceding window of blocks. The energy of a block is used to detect a beat. If the
            energy is above a certain threshold then the block is considered to contain a beat. The threshold is defined
            starting from the average energy of the window of blocks preceding the one we are analyzing.\\
            If a block j is made of 1024 samples and the song is stereo, its energy can be computed as:
            \[E_j = \sum_{i=0}^{1023}left[i]^2 + right[i]^2  \]
            This technique was deemed imprecise by the article.


    \section{Terms}
        If any are left blank go look em up!\\ \\
         \textbf{Sampling Frequency} - the number of samples per second in a Sound \\ \\
         \textbf{Discrete wavelet transform} - \\ \\
         \textbf{Low-pass} - Low-pass filters pass through frequencies below their cutoff frequencies, and progressively
          attenuates frequencies above the cutoff frequency. Low-pass filters are used in audio crossovers to remove
          high-frequency content from signals being sent to a low-frequency subwoofer system.\\ \\
         \textbf{High-pass} - A high-pass filter does the opposite, passing high frequencies above the cutoff frequency,
          and progressively attenuating frequencies below the cutoff frequency. A high-pass filter can be used in an
          audio crossover to remove low-frequency content from a signal being sent to a tweeter.\\ \\
         \textbf{Bandpass} - A bandpass filter passes frequencies between its two cutoff frequencies, while attenuating
          those outside the range. A band-reject filter attenuates frequencies between its two cutoff frequencies,
          while passing those outside the 'reject' range.\\ \\
         \textbf{All-pass} - An all-pass filter passes all frequencies, but affects the phase of any given sinusoidal
         component according to its frequency.\\ \\
         \textbf{attack} -  is the time interval during which the amplitude envelope increases\\ \\
         \textbf{transient} -  \\ \\
         \textbf{onset} - a single instant chosen to mark the temporally extended transient. \\ \\
         \textbf{additive} -  \\ \\
         \textbf{oscillatory} -  \\ \\
         \textbf{} -  \\ \\
         \textbf{} -  \\ \\

    \section{Links to research}
        \url{http://mziccard.me/2015/05/28/beats-detection-algorithms-1/}\\
        \url{http://archive.gamedev.net/archive/reference/programming/features/beatdetection/index.html}\\
        \url{http://shepazu.github.io/Audio-EQ-Cookbook/audio-eq-cookbook.html}\\
        \url{http://citeseerx.ist.psu.edu/viewdoc/download?doi=10.1.1.332.989&rep=rep1&type=pdf}\\
        \url{}\\
        \url{}\\
\end{document}